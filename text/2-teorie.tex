\chapter{Teoretický základ}
\label{2-teorie}
\section{Eroze}
\subsection{Co je eroze?}
Eroze, z latinského „erodere“ – rozhlodávat, je přirozeným přírodním procesem, během kterého dochází mechanickým působením vnějších faktorů – vody, větru a sněhu k rozrušování půdního povrchu, transportu půdních částic a jejich ukládání na novém místě (sedimentaci).

Vliv přirozené neboli geologické eroze problémem, a naopak je jedním ze základních procesů tvorby krajiny po milióny let. Dochází při ní k uvolňování biogenních prvků (např. fosfor, draslík, síra), jež jsou nezbytné pro život všech organismů a jinak by zůstaly navázané v horninách. Rychlost geologické eroze je srovnatelná s rychlostí, kterou je přirozenými procesy vytvářena půda nová.

Bohužel, ale vlivem antropogenních jevů nastává tzv. zrychlená půdní eroze, která má negativní vliv jak na kvalitu půd (zúžení půdního profilu, snížení obsahu živin, zvýšení štěrkovitosti,), jelikož při ní dochází k odstraňování nejúrodnější složky půdy – ornice, tak na plodiny na ní pěstované (mechanické poškození, ztráty osiva a hnojiv). Kvůli čemuž může na silně erodovaných pozemcích docházet ke ztrátám až 3/4 z hektarového výnosu. Následná sedimentace půdních částic zanáší vodní toky, nádrže a při zvýšeném povodňovém průtoku může způsobovat rozsáhlé škody v intravilánech obcí.
\subsection{Rozdělení eroze}
Kromě rozdělení eroze dle intenzity na přirozenou a zrychlenou, je možné erozi dělit podle činitele, jenž jí způsobuje na erozi vodní, větrnou a sněhovou.
\paragraph{Vodní eroze}
Při vodní (akvatické) erozi dochází k rozrušování zemského povrchu dopadem vodních kapek, nejsilněji při letních přívalových deštích, kdy se dopadající voda nestihne vsáknout do půdy a následný povrchový odtok způsobuje další vymílání a transport půdních částic. Vodní eroze je v České republice nejčastější a je jí ohroženo více než 50 \% veškeré orné půdy. Je ovlivněna především sklonem pozemku, délkou pozemku po spádnici, vegetačním pokryvem a strukturou půdy.

Vodní erozi můžeme dále dělit dle formy odtoku. Jako první probíhá při dešti plošná eroze (plošný splach), ta je charakteristická rozrušováním a smyvem půdní hmoty z celého území. Jejím prvním stupněm je eroze selektivní, při níž jsou povrchovým odtokem odnášeny jemné půdní částice, na které jsou vázány chemické látky. Tím je způsobena větší hrubozrnnost a snížení obsahu živin v půdě zasažené erozí. Naopak půda obohacená smyvem je jemnozrnnější a bohatší na živiny. Druhý stupeň plošné eroze probíhá při větší kinetické energii povrchově stékající vody a střídání málo odolných a odolných vrstev, odtud také název – eroze vrstevná. Projevuje se po celé ploše svahu a odnáší celé málo odolné vrstvy ornice.

Druhou formou odtoku je výmolová eroze, vznikající postupným soustřeďováním vody stékající po povrchu a jejím zařezáváním do půdy. Prvním stádiem je eroze rýžková či brázdová, kdy na povrchu vznikají úzké rýžky a mělčí širší brázdy, které tvoří na postiženém území hustou síť. Voda odtékající v soustředěném odtoku má větší kinetickou energii, čímž transportuje půdu ve větším rozsahu. Důsledkem stékání rýžek a brázd voda dále získává na síle, vzniká rýhová eroze. Pokud zesílí mění se v erozi výmolovou či devastující erozi stržovou, které způsobují hluboké výmoly a trže.
\paragraph{Větrná eroze}
Druhým nejvýznamnějším typem v ČR je eroze větrná neboli eolická, jež ohrožuje téměř 10 \% výměry orné půdy. Jedná se území, kde se vyskytují výsušné větry, pozemky jsou zde sceleny do obrovských jednotek, na kterých se hospodaří monokulturně a vlivu mechanické síly větru (abrazi) není bráněno větrolamy, např. oblasti jižní Moravy a Polabí. 

Při větrné erozi dochází k unášení půdních částic různých velikostí, od velmi jemných zrnek (<0,1 mm), která jsou transportována velmi snadno ve formě suspenze, často i na velké vzdálenosti a ve větším množství mohou způsobovat písečné bouře a zákaly atmosféry. Přes částice střední velikosti (0,1-0,4mm), která představují 50-80\% celkového objemu přenášeného větrem a způsobují největší škody kolizí a rozrušováním vytvořeného půdního agregátu. Po největší zrna (0,5-2 mm), která se pohybují sunutím po povrchu a představují asi 25\% objemu větrné eroze.

\paragraph{Sněhová eroze}
Sněhová eroze transportuje půdní částice obdobně jako eroze vodní, avšak vyznačuje se určitými specifiky, jedním z nich je zanedbatelný vliv kinetické energie dopadajících sněhových vloček na rozrušování povrchu půdy, erozní účinnost sněhu je tedy výhradně při jeho tání. 

\subsection{Protierozní opatření}
Pojem protierozní opatření označuje konkrétní kroky vedoucí k minimalizaci negativních dopadů eroze na půdu, vodní toky a nádrže, intravilán obcí apod. Protierozní opatření dělíme na organizační, tj. vhodné umístění pěstovaných plodin, pásové střídání plodin nebo návrhy vegetačních pásů mezi pozemky, agrotechnická (půdoochranné obdělávání) a technická (příkopy, terasy, protierozní nádrže a další). Ve většině případů jde o komplex těchto opatření, která se vzájemně doplňují, aby byly splněny všechny požadavky na ně kladené a zároveň byly respektovány potřeby zemědělské výroby.

\paragraph{Organizační protierozní opatření}
Základní organizační protierozní opatření je navržení vhodné velikosti a tvaru pozemku a jeho situování delší stranou po vrstevnicích. Dodržení tohoto základního pravidla, není vždy jednoduché, neboť proti sobě působí dva faktory – přírodní, který vede k vytváření menších pozemků s lepší erozní a ekologickou stabilitou, a faktor ekonomický, jenž naopak upřednostňuje tvorbu dostatečně velkých pozemků pro efektivní využívání zemědělských strojů. Tvar a velikost pozemku je tedy kompromisem mezi geografickými podmínkami a požadavky vlastníků (uživatelů) půdy. Obecně se doporučuje navrhování půdních bloků do 50 ha na rovinných územích a do 20 ha v územích členitějších.

Dalším krokem je delimitace (vymezení hranic) druhu pozemků, která slouží jako funkční a prostorová optimalizace využití pozemků zemědělského půdního fondu. Ten je rozdělen na ornou půdu, zahrady, louky, pastviny, vinice, sady a chmelnice. K rozdělení dochází na základě erozní ohroženosti pozemků, kdy nejohroženější pozemky, které nemohou být dále využívány jako orná půda, jsou zatravněny. Umístěním trvalého travního porostu by měly být chráněny také průlehy, ochranné hrázky a břehy vodních toků a nádrží. Je vhodné umístění liniových pásů travního porostu v drahách výmolové eroze. Nejúčinnější ochrany proti vodní erozi je dosaženo zalesněním území, optimálně lesem smíšeným.

Důležitým faktorem je tedy přítomnost vegetačního pokryvu, jehož význam je nejvyšší v období letních přívalových dešťů. Vegetace chrání půdu před přímým dopadem dešťových kapek, podporuje vsak vody do půdy a její kořenový systém pomáhá zvyšovat soudržnost půdy vůči stékající vodě. Těchto znalostí se využívá při výběru dalších organizačních opatření, kterými mohou být např. včasný výsev plodin, posunutí podmítky do období s nižším výskytem přívalových dešťů (tj. na září), zařazení bezorebně setých meziplodin a v neposlední řadě rozmístění plodin dle ohroženosti pozemku, kdy pozemky rovinné nebo mírně sklonité je vhodné využít pro pěstování plodin nedostatečně chránících půdu před erozí (širokořádkové plodiny) a pro pozemky více ohrožené volit plodiny s lepším protierozním charakterem nebo pásové střídání plodin, kdy se střídají pásy plodin chránících půdu (např. travní porost, vojtěška, jetel) s plodinami s nižší protierozní účinností (okopaniny, kukuřice).

\paragraph{Agrotechnická protierozní opatření}
Jelikož nejvíce ohrožena erozí je půda bez vegetačního pokryvu, snaží se agrotechnická protierozní opatření zkrátit období, kdy je půda obnažena na minimum. Nejrizikovějšími obdobími jsou zejména období tání sněhu a letních přívalových dešťů. Úspěšným prostředkem je cílené využívání biomasy z meziplodin a posklizňových zbytků, přičemž je třeba dbát na neomezování možnosti infiltrace vody do půdy. 

Mezi nejúčinnější způsoby patří technologie ochranného zpracování půdy, kdy je místo orby využíváno kypření půdy bez obracení zpracovávané vrstvy půdy. Ovšem i při orbě lze snížit vliv vodní eroze dodržením pravidla o orání ve směru vrstevnic, kdy brázdy mohou omezit povrchový odtok vody. Dále existuje množství dalších technologií, jež jsou specifické pro pěstované plodiny, jejich přehled je možné najít v metodice Ochrana zemědělské půdy před erozí (str. 60-70).

\paragraph{Technická protierozní opatření}
Technické prvky jsou základem protierozní ochrany, a to především tam, kde smyv z pozemků má důsledky pro intravilán obce. Jedná se o převážně liniové prvky, které jsou optimálně rozmisťovány tak, aby došlo k přerušení délky svahu a rozčlenění pozemků. Dále mají vliv na agrotechnické opatření, kdy usměrňují směr obdělávání pozemků a způsob hospodaření jejich uživatelů. Vhodná je i jejich kombinace s organizačními opatřeními, kdy po rozdělení svahu liniovými opatřeními jsou do nově vymezených pásů umístěny plodiny s různou protierozním povahou. V neposlední řadě je jejich funkce i ekologického a krajině estetického charakteru. Technická protierozní opatření jsou navrhována během pozemkových úprav a patří mezi ně průlehy, příkopy, hrázky, meze, ochranné nádrže, stabilizování drah soustředného odtoku a terasování. Způsoby navrhování zmiňovaných opatření lze dohledat v metodice Ochrana zemědělské půdy před erozí (str. 70-90).

\section{Historie}
\subsection{Počátky výzkumu eroze}
Výzkumem eroze se začali zabývat vědci v USA začátkem 20. století, jednou z významných osobností při prvních krocích výzkumu byl Hugh Bennett, který začal pozoroval zvýšenou erozi na zemědělské půdě a její vliv na výnos. Bennett na tento rostoucí problém veřejně upozorňoval, do většího povědomí se však ochrana zemědělské půdy dostala až na přelomu 20. a 30. let v souvislosti s mohutnými prachovými bouřemi (tzv. Dust Bowl, Dirty Thirties).

\paragraph{Dust Bowl}, slovní spojení, jež bylo původně použito pro oblast vzniku prachových bouří -  Velkých planin a následně se stalo synonymem i pro období 30. let (proto také druhý název Dirty Thirties), kdy se bouře objevovaly. Příčinou byla rychlá změna suchých prérií s nízkým srážkovým úhrnem (místy pod 250 mm/rok) na zemědělskou půdu. Tuto změnu umožnilo rozšíření zemědělských strojů, zejména traktorů, díky kterým mohla být na rozsáhlém území využita hluboká orba, kterou byla zničena původní vegetace chránící půdní povrch. Místo původních travin byla na většině území vysazena pšenice, přičemž nebylo dodržováno ani střídavé hospodaření. 

Ve spojení s nulovou ochranou půdy došlo tímto způsobem zemědělství v obdobích sucha k proměně půdy v prach, který byl unášen větrem a vytvářel mohutná prachová oblaka devastující krajinu. Erozí zasáhla okolo 100 milionů akrů půdy (v roce 1934 bylo bouří zasaženo dokonce východní pobřeží USA včetně New Yorku), zanechala bez domova téměř 500 tisíc lidí a způsobila stěhování až 3,5 milionu lidí ze zasažené oblasti, čímž se jednalo o největší migrační vlnu v historii USA. Touto katastrofou byla dále prohloubena Velká hospodářská krize.

První reakcí bylo v roce 1929 zřízení deseti výzkumných stanic, ve kterých byla na tzv. jednotkových pozemcích (délka 22,13 m, sklon 9$\%$, trvalý úhor, obdělávání ve směru sklonu) pozorována eroze a měřen smyv. Z měření byla regresní analýzou odvozeny faktory ovlivňující erozi a účinnost ochranných opatření. Další krok učinil v roce 1932 nově zvolený americký prezident Franklin Roosevelt, který v rámci opatření proti trvající hospodářské krizi zřídil úřad zabývající se půdní erozí (SES – Soil Erosion Service), v čele kterého stanul H. Bennett. Tento úřad začal učit farmáře šetrnějším metodám hospodaření, budovat velké množství protierozních opatření, při čemž zaměstnával dělníky zasažené ekonomickou depresí, a samozřejmě pokračoval ve výzkumu eroze.

\subsection{Vývoj USLE}
Díky studiím z výzkumných stanic, které postupně vznikaly, bylo shromážděno velké množství informací, které byly dále využity pro výpočty erozního smyvu. Vývoj matematických rovnic pro odhad množství půdní eroze a vliv ochranných opatření začal v druhé polovině 30. let. První empirický model pro odhad průměrné roční ztráty půdy způsobené vodní erozí publikoval v roce 1940 A. W. Zing. Jeho rovnice zahrnovala vliv sklonu pozemku a jeho délku. Konstanty byly určovány pro kukuřičný pás (Corn Belt), hlavní zemědělskou oblast USA.
\begin{align}
   \label{zing1940} G=L^{0,6}\cdot S^{1,4}\cdot C
\end{align}
\hspace*{2cm}$G \cdots$ průměrná roční ztráta půdy\\
\hspace*{2cm}$L \cdots$ délka svahu \\ 
\hspace*{2cm}$S \cdots$ sklon svahu \\ 
\hspace*{2cm}$C \cdots$ koeficient zahrnující ostatní faktory\\

Tento model v roce 1941 rozšířil D. D. Smith, který do rovnice zahrnul vliv technických protierozních opatření. Smith jako první využil její výsledky využil k určení maximální délky svahu na specifickém území a stanovení metod ochrany.
\begin{align}
   \label{smith1941} G=L^{\frac{3}{5}}\cdot S^{\frac{7}{5}}\cdot P\cdot C
\end{align}
\hspace*{2cm}$G \cdots$ průměrná roční ztráta půdy\\
\hspace*{2cm}$L \cdots$ délka svahu \\ 
\hspace*{2cm}$S \cdots$ sklon svahu \\ 
\hspace*{2cm}$P \cdots$ faktor protierozních opatření \\
\hspace*{2cm}$C \cdots$ koeficient zahrnující ostatní faktory \\

Tato rovnice byla G. M. Browning použita pro sestavení empirického modelu pro stát Iowa, kde jsou poprvé zahrnuty geologické faktory půdy a agrotechnických opatření.
\begin{align}
   \label{browning} G=10\cdot\left( K^{\prime}\cdot O^{\prime}\cdot  L^{\prime}\cdot S^{\prime}\cdot C^{\prime}\cdot P^{\prime} \right)
\end{align}
\hspace*{2cm}$G \cdots$ průměrná roční ztráta půdy\\
\hspace*{2cm}$K^{\prime} \cdots$ faktor erodovatelnosti půdy \\ 
\hspace*{2cm}$O^{\prime} \cdots$ faktor geologického podkladu \\  
\hspace*{2cm}$L^{\prime} \cdots$ délka svahu \\ 
\hspace*{2cm}$S^{\prime} \cdots$ sklon svahu \\ 
\hspace*{2cm}$C^{\prime} \cdots$ faktor vegetačního pokryvu \\
\hspace*{2cm}$P^{\prime} \cdots$ faktor protierozních opatření \\

V roce 1947 Musgrave publikoval první erozní model, který zahrnoval vliv přívalového deště. V této rovnici byly také překlasifikovány hodnoty faktorů pro kukuřičný pás.

\begin{align}
   \label{musgrave1947} G=K\cdot C\cdot L^{0,35}\cdot S^{1,35}\cdot R^{1,75}
\end{align}
\hspace*{2cm}$G \cdots$ průměrná roční ztráta půdy\\
\hspace*{2cm}$K \cdots$ geologický faktor eroze půdy \\ 
\hspace*{2cm}$C \cdots$ faktor vegetačního pokryvu \\
\hspace*{2cm}$L \cdots$ délka svahu \\ 
\hspace*{2cm}$S \cdots$ sklon svahu \\  
\hspace*{2cm}$R \cdots$ úhrn deště s periodicitou 0,5 za 30 minut \\

V roce 1954 na Purdue University, která měla v té době vedoucí postavení v oblasti výpočetních technologií, vzniklo pod vedením W. H. Wischmeier centrum pro shromažďování dat o půdní erozi (Soil Loss Data Center). Přístup k nejmodernějším technologiím dovolil zrychlit analyzování velkého množství dat (v letech 1940-1965 přes 10~000 zmonitorovaných smyvů ročně), jež byly shromažďovány od 30. let. Výsledkem byla v roce 1965 první kompletní publikace vytvořená týmem pod vedením Wischmeier a Smith, která byla ještě zpřesněna dalším výzkumem a upravena do současné formy publikované v roce 1978. 

\begin{align}
   \label{usle1978} G=R\cdot K\cdot L\cdot S\cdot C\cdot P
\end{align}
\hspace*{2cm}$G \cdots$ průměrná roční ztráta půdy $\left( t\cdot ha^{-1}\cdot rok^{-1} \right)$\\
\hspace*{2cm}$R \cdots$ faktor erozní účinnosti deště \\
\hspace*{2cm}$K \cdots$ geologický faktor eroze půdy \\ 
\hspace*{2cm}$L \cdots$ faktor délky svahu \\ 
\hspace*{2cm}$S \cdots$ faktor sklonu svahu \\  
\hspace*{2cm}$C \cdots$ faktor vegetačního pokryvu \\
\hspace*{2cm}$C \cdots$ faktor protierozních opatření \\

\subsection{Modifikace USLE}
\paragraph{RUSLE}
V roce 1997 byla publikována Revidovaná univerzální rovnice ztráty půdy (Revised Universal Soil Loss Equation), ta byla odvozena na základě revize, prověření a aktualizace USLE. Došlo zde k významné změně způsobu určení erozních faktorů (např. zpřesnění časového průběhu hodnot R faktoru v patnáctidenním intervalu, zpřesnění časového průběhu K faktoru v důsledku zhutňování povrchu a rozpadu půdních agregátů srážkami či obhospodařováním, nové vztahy pro LS faktor). Výhodou využití RUSLE je výpočet pomocí volně přístupného programu pro DOS nebo Windows, jehož výsledkem je přesnější určení erozního modelu. Nevýhodou je nutnost většího množství vstupních dat.
\paragraph{MUSLE}
Druhou významnou úpravou je Modifikovaná univerzální rovnice ztráty půdy (Modified Universal Soil Loss Equation, která zahrnuje transportní činitele během erozního procesu a stanovuje množství splavenin z přívalového deště v povodí do velikosti 15 km$^{2}$.

\subsection{Současný stav}
USLE je v současnosti používána pro výpočet dlouhodobé eroze ve většině států světa a jedinou doporučenou metodou v ČR (Metodika VÚMOP), existují k ní rozsáhlá katalogová data a je i poměrně jednoduchá. Ovšem i přes úpravy a postupné zlepšení schopnosti odhadu dlouhodobé ztráty půdy je USLE i další zmíněné metody empirické, tedy založené na využití koeficientů získaných při pozorování v terénu (na jednotkových pozemcích). Jejich přesnost je tedy závislá na přesnosti klasifikace jednotlivých faktorů pro specifickou situaci. 

V posledních letech je snaha omezit zavedený empirický základ při posuzování erozních procesů a nahradit ho kvalitnějšími simulačními metodami. Cílem této změny je hodnotit důsledky eroze nejen vůči zemědělské půdě, ale i vůči jiným ekologickým celkům (např. vodní toky, nádrže) a také řešení eroze na kratším časovém horizontu.  K vývoji a zdokonalování simulačních modelů přispívá rozvoj výpočetní techniky, geografických informačních systémů a rozšiřování znalostí v dané oblasti.

\section{Univerzální rovnice ztráty půdy}
\subsection{Faktor účinnosti deště (R)}
\subsection{Faktor struktury půdy (K)}
\subsection{Faktor délky a sklonu svahu (L, S)}
\subsection{Ochranný vliv vegetace (C)}
\subsection{Faktor protierozních opatření (P)}


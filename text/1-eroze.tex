
%!TEX ROOT=bp-novotny-text.tex
\part{Teoretická část}

\chapter{Eroze}
\section{Co je eroze?}
Eroze, z latinského „erodere“ – rozhlodávat, je přirozeným přírodním procesem, během kterého dochází mechanickým působením vnějších faktorů – vody, větru a sněhu k rozrušování půdního povrchu, transportu půdních částic a jejich ukládání na novém místě (sedimentaci).

Vliv přirozené neboli geologické eroze problémem, a naopak je jedním ze základních procesů tvorby krajiny po milióny let. Dochází při ní k uvolňování biogenních prvků (např. fosfor, draslík, síra), jež jsou nezbytné pro život všech organismů a jinak by zůstaly navázané v horninách. Rychlost geologické eroze je srovnatelná s rychlostí, kterou je přirozenými procesy vytvářena půda nová.

Bohužel, ale vlivem antropogenních jevů nastává tzv. zrychlená půdní eroze, která má negativní vliv jak na kvalitu půd (zúžení půdního profilu, snížení obsahu živin, zvýšení štěrkovitosti,), jelikož při ní dochází k odstraňování nejúrodnější složky půdy – ornice, tak na plodiny na ní pěstované (mechanické poškození, ztráty osiva a hnojiv). Kvůli čemuž může na silně erodovaných pozemcích docházet ke ztrátám až 3/4 z hektarového výnosu. Následná sedimentace půdních částic zanáší vodní toky, nádrže a při zvýšeném povodňovém průtoku může způsobovat rozsáhlé škody v intravilánech obcí.
\section{Rozdělení eroze}
Kromě rozdělení eroze dle intenzity na přirozenou a zrychlenou, je možné erozi dělit podle činitele, jenž jí způsobuje na erozi vodní, větrnou a sněhovou.
\subsection{Vodní eroze}
Při vodní (akvatické) erozi dochází k rozrušování zemského povrchu dopadem vodních kapek, nejsilněji při letních přívalových deštích, kdy se dopadající voda nestihne vsáknout do půdy a následný povrchový odtok způsobuje další vymílání a transport půdních částic. Vodní eroze je v České republice nejčastější a je jí ohroženo více než 50 \% veškeré orné půdy. Je ovlivněna především sklonem pozemku, délkou pozemku po spádnici, vegetačním pokryvem a strukturou půdy.

Vodní erozi můžeme dále dělit dle formy odtoku. Jako první probíhá při dešti plošná eroze (plošný splach), ta je charakteristická rozrušováním a smyvem půdní hmoty z celého území. Jejím prvním stupněm je eroze selektivní, při níž jsou povrchovým odtokem odnášeny jemné půdní častice, na které jsou vázány chemické látky. Tím je způsobena větší hrubozrnnost a snížení obsahu živin v půdě zasažené erozí. Naopak půda obohacená smyvem je jemnozrnnější a bohatší na živiny. Druhý stupeň plošné eroze probíhá při větší kinetické energii povrchově stékající vody a střídání málo odolných a odolných vrstev, odtud také název – eroze vrstevná. Projevuje se po celé ploše svahu a odnáší celé málo odolné vrstvy ornice.

Druhou formou odtoku je výmolová eroze, vznikající postupným soustřeďováním vody stékající po povrchu a jejím zařezáváním do půdy. Prvním stádiem je eroze rýžková či brázdová, kdy na povrchu vznikají úzké rýžky a mělčí širší brázdy, které tvoří na postiženém území hustou síť. Voda odtékající v soustředěném odtoku má větší kinetickou energii, čímž transportuje půdu ve větším rozsahu. Důsledkem stékání rýžek a brázd voda dále získává na síle, vzniká rýhová eroze. Pokud zesílí mění se v erozi výmolovou či devastující erozi stržovou, které způsobují hluboké výmoly a trže.
\subsection{Větrná eroze}
Druhým nejvýznamnějším typem v ČR je eroze větrná neboli eolická, jež ohrožuje téměř 10 \% výměry orné půdy. Jedná se území, kde se vyskytují výsušné větry, pozemky jsou zde sceleny do obrovských jednotek, na kterých se hospodaří monokulturně a vlivu mechanické síly větru (abrazi) není bráněno větrolamy, např. oblasti jižní Moravy a Polabí. Při větrné erozi dochází k unášení půdních částic různých velikostí, od velmi jemných zrnek (<0,1 mm), která jsou transportována velmi snadno ve formě suspenze, často i na velké vzdálenosti a ve větším množství mohou způsobovat písečné bouře a zákaly atmosféry. Přes částice střední velikosti (0,1-0,4mm), která představují 50-80\% celkového objemu přenášeného větrem a způsobují největší škody kolizí a rozrušováním vytvořeného půdního agregátu. Po největší zrna (0,5-2mm), která se pohybují sunutím po povrchu a představují asi 25\% objemu větrné eroze.

\subsection{Sněhová eroze}
Sněhová eroze transportuje půdní částice obdobně jako eroze vodní, avšak vyznačuje se určitými specifiky, jedním z nich je zanedbatelný vliv kinetické energie dopadajících sněhových vloček na rozrušování povrchu půdy, erozní účinnost sněhu je tedy výhradně při jeho tání. 

\section{Protierozní opatření}
Pojem protierozní opatření označuje konkrétní kroky vedoucí k minimalizaci negativních dopadů eroze na půdu, vodní toky a nádrže, intravilán obcí apod. Protierozní opatření dělíme na organizační, tj. vhodné umístění pěstovaných plodin, pásové střídání plodin nebo návrhy vegetačních pásů mezi pozemky, agrotechnická (půdoochranné obdělávání) a technická (příkopy, terasy, protierozní nádrže a další). Ve většině případů jde o komplex těchto opatření, která se vzájemně doplňují, aby byly splněny všechny požadavky na ně kladené a zároveň byly respektovány potřeby zemědělské výroby.

\subsection{Organizační protierozní opatření}
Základní organizační protierozní opatření je navržení vhodné velikosti a tvaru pozemku a jeho situování delší stranou po vrstevnicích. Dodržení tohoto základního pravidla, není vždy jednoduché, neboť proti sobě působí dva faktory – přírodní, který vede k vytváření menších pozemků s lepší erozní a ekologickou stabilitou, a faktor ekonomický, jenž naopak upřednostňuje tvorbu dostatečně velkých pozemků pro efektivní využívání zemědělských strojů. Tvar a velikost pozemku je tedy kompromisem mezi geografickými podmínkami a požadavky vlastníků (uživatelů) půdy. Obecně se doporučuje navrhování půdních bloků do 50 ha na rovinných územích a do 20 ha v územích členitějších.

Dalším krokem je delimitace (vymezení hranic) druhu pozemků, která slouží jako funkční a prostorová optimalizace využití pozemků zemědělského půdního fondu. Ten je rozdělen na ornou půdu, zahrady, louky, pastviny, vinice, sady a chmelnice. K rozdělení dochází na základě erozní ohroženosti pozemků, kdy nejohroženější pozemky, které nemohou být dále využívány jako orná půda, jsou zatravněny. Umístěním trvalého travního porostu by měly být chráněny také průlehy, ochranné hrázky a břehy vodních toků a nádrží. Je vhodné umístění liniových pásů travního porostu v drahách výmolové eroze. Nejúčinnější ochrany proti vodní erozi je dosaženo zalesněním území, optimálně lesem smíšeným.

Důležitým faktorem je tedy přítomnost vegetačního pokryvu, jehož význam je nejvyšší v období letních přívalových dešťů. Vegetace chrání půdu před přímým dopadem dešťových kapek, podporuje vsak vody do půdy a její kořenový systém pomáhá zvyšovat soudržnost půdy vůči stékající vodě. Těchto znalostí se využívá při výběru dalších organizačních opatření, kterými mohou být např. včasný výsev plodin, posunutí podmítky do období s nižším výskytem přívalových dešťů (tj. na září), zařazení bezorebně setých meziplodin a v neposlední řadě rozmístění plodin dle ohroženosti pozemku, kdy pozemky rovinné nebo mírně sklonité je vhodné využít pro pěstování plodin nedostatečně chránících půdu před erozí (širokořádkové plodiny) a pro pozemky více ohrožené volit plodiny s lepším protierozním charakterem nebo pásové střídání plodin, kdy se střídají pásy plodin chránících půdu (např. travní porost, vojtěška, jetel) s plodinami s nižší protierozní účinností (okopaniny, kukuřice).

\subsection{Agrotechnická protierozní opatření}
Jelikož nejvíce ohrožena erozí je půda bez vegetačního pokryvu, snaží se agrotechnická protierozní opatření zkrátit období, kdy je půda obnažena na minimum. Nejrizikovějšími obdobími jsou zejména období tání sněhu a letních přívalových dešťů. Úspěšným prostředkem je cílené využívání biomasy z meziplodin a posklizňových zbytků, přičemž je třeba dbát na neomezování možnosti infiltrace vody do půdy. 

Mezi nejúčinnější způsoby patří technologie ochranného zpracování půdy, kdy je místo orby využíváno kypření půdy bez obracení zpracovávané vrstvy půdy. Ovšem i při orbě lze snížit vliv vodní eroze dodržením pravidla o orání ve směru vrstevnic, kdy brázdy mohou omezit povrchový odtok vody. Dále existuje množství dalších technologií, jež jsou specifické pro pěstované plodiny, jejich přehled je možné najít v metodice Ochrana zemědělské půdy před erozí (str. 60-70).

\subsection{Technická protierozní opatření}
Technické prvky jsou základem protierozní ochrany, a to především tam, kde smyv z pozemků má důsledky pro intravilán obce. Jedná se o převážně liniové prvky, které jsou optimálně rozmisťovány tak, aby došlo k přerušení délky svahu a rozčlenění pozemků. Dále mají vliv na agrotechnické opatření, kdy usměrňují směr obdělávání pozemků a způsob hospodaření jejich uživatelů. Vhodná je i jejich kombinace s organizačními opatřeními, kdy po rozdělení svahu liniovými opatřeními jsou do nově vymezených pásů umístěny plodiny s různou protierozním povahou. V neposlední řadě je jejich funkce i ekologického a krajině estetického charakteru. Technická protierozní opatření jsou navrhována během pozemkových úprav a patří mezi ně průlehy, příkopy, hrázky, meze, ochranné nádrže, stabilizování drah soustředného odtoku a terasování. Způsoby navrhování zmiňovaných opatření lze dohledat v metodice Ochrana zemědělské půdy před erozí (str. 70-90).

\section{Historie}

\chapter{Univerzální rovnice ztráty půdy}
\section{Faktor délky a sklonu svahu (L, S)}
\section{Faktor struktury půdy (K)}
\section{Ochranný vliv vegetace (C)}
\section{Faktor účinnosti deště (R)}
\section{Faktor protierozních opatření (P)}


%!TEX ROOT=radek-novotny-bp-2017.tex
\chapter{Úvod}

Ačkoliv na první pohled se eroze v našich podmínkách nemusí zdát jako
vážný problém a v měřítku eroze k níž dochází například v rozvojových
zemích Afriky by se mohla zdát až zanedbatelná, je tomu právě
naopak. V České republice je v současnosti erozně ohrožena více než
polovina zemědělské orné půdy, z které tímto procesem nenávratně mizí
nejúrodnější vrstva – ornice, jež vznikala po stovky let.

Tento stav nastal v ČR důsledkem postupných změn v zemědělském
hospodaření. První zásadní změnou byla kolektivizace zemědělství a ní
související zcelování polních jednotek, přičemž docházelo k
odstraňování mezí a remízků, jež tvořily přirozenou protierozní
ochranu. Bohužel ani po navrácení pozemků do soukromého vlastnictví
nedošlo k očekávanému rozdělení velkých polních jednotek a obnovení
přirozené protierozní ochrany.

Druhá změna značně ovlivňující erozi nastala v devadesátých letech,
kdy ubývala živočišná výroba a s ní i plodiny pro ni pěstované –
vojtěška a jeteloviny, které mají na půdu velmi dobrý protierozní
vliv. Ty byly díky štědré dotační politice nahrazeny plodinami
sloužícími zejména pro výrobu biopaliv – řepkou a kukuřicí, jejichž
protierozní charakter je naopak nevalný.

V posledních letech poutají projevující se důsledky těchto změn k
problematice eroze větší pozornost, se kterou přichází i zvýšená snaha
problém řešit. Příkladem může být Redesign erozní ohroženosti půdy v
LPIS (č. j.: 43526/2016-MZE) schválený v loňském roce na Ministerstvu
zemědělství. Tento materiál postupně snižuje hodnoty přípustné ztráty
půdy erozí (Gp), důsledkem čehož by mělo být v roce 2030 chráněno až
60\% orné půdy.

Pro návrh protierozních opatření je nutné určení ohrožených pozemků,
které probíhá stanovením průměrné ztráty půdy (G) a následným
porovnáním se zmíněnou přípustnou hodnotou (Gp). K tomuto určení se v
ČR využívá rovnice USLE, která je po rozvoji GIS v posledních letech
softwarově implementována. Nejpoužívanějším nástrojem je v současnosti
aplikace Atlas EROZE, která nabízí kompletní výpočet se zahrnutím
všech faktorů ovlivňujících erozi.

Tato aplikace ovšem není volně dostupná a uživatel využívající open
source nástroje je odkázán na vlastní ruční řešení např. v softwaru
GRASS, který nástroje pro dané kroky výpočtu obsahuje.

Cílem této práce je tedy vytvoření open source nástroje, který výpočet
erozní ohroženosti provede automaticky, přičemž od uživatele se
očekává pouze zadání vstupních dat.
